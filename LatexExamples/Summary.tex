\chapter{Zusammenfassung und Ausblick}
Kapitel 1 befasste sich mit den Möglichkeiten der computergestützten Bilderkennung und ihrer Notwendigkeit unter anderem bei der automatisierten Auswertung von visuellen Daten und der Materialprüfung. Dabei wurde die Bedeutung einer korrekten Segmentierung der Bildinhalte zu deren korrekter Erfassung ebenso verdeutlicht wie die Vorteile und Probleme der verschiedenen Segmentierungsmethoden.
Eine automatische Durchführung, möglichst ohne Einschränkungen des Einsatzgebietes, stand im Vordergrund der Betrachtungen, weshalb eine plattformunabhängige Implementation in Java erfolgte.
\\ \\   
Im Kapitel 2 wurden notwendige mathematische und formale Grundlagen gelegt. Ebenso schloss sich die Einführung in die prinzipielle Funktionsweise der verwendeten Algorithmen K-Means und der Partikel Schwarm Optimierung sowie die zur Beschleunigung vorgesehene Pyramide an.
\\ \\  
Aufbauend darauf behandelte Kapitel 3 die Erläuterung der Verfahrensstruktur und eventuell auftretender Effekte bei der Anwendung sowie eine Beschreibung wichtiger Einflussfaktoren der Algorithmen, bevor sich in Kapitel 4 die Evaluation des Verfahrens vollzog.
\\ \\
Soll das Verfahren zum realen Einsatz gelangen, so ist eine Geschwindigkeitsverbesserung bei der Erstellung des Merkmalsraumes ebenso anzuraten wie eine aufgabenspezifischere Einstellung der Parameter für die Partikel Schwarm Optimierung, insbesondere von $c_1$, $c_2$, $r$ und $n$.

