%======================================================================
%	Metadata
%======================================================================
%	$Id$
%	Matthias Kupfer
%======================================================================

\newcommand{\dcsubject}{Master Thesis}
% z.B. (Diplom/Studien/Haus)arbeit, Praktikumsbericht, Studie, Beleg, 
% (Pro/Haupt/Ober)seminar, Seminar usw.
\newcommand{\dctitle}{System Design, Implementation and Evaluation of a User-centered Application for Indoor Sound Classification}
\newcommand{\dcsubtitle}{~} % Subtitle, if necessary

\newcommand{\dcauthorlastname}{Sun}
\newcommand{\dcauthorfirstname}{Maoyin}
\newcommand{\dcauthoremail}{maoyin.sun@s2016.tu-chemnitz.de}
\newdateformat{mydate}{{\THEDAY} \monthname[\THEMONTH] \THEYEAR}
\newcommand{\dcdate}{\mydate\today}
%\newcommand{\dcdate}{\today}

\newcommand{\dcplace}{Chemnitz} % Ort, kann an der TU meist so bleiben
\newcommand{\dcuni}{Technische Universität \dcplace}
\newcommand{\dcdepart}{Faculty of Computer Science} % Fakultätsangabe
\newcommand{\dcprof}{Junior Professorship of Media Computing} % Angabe der Professur

\newcommand{\dcpruefer}{Jun.-Prof. Dr. rer. nat. Danny Kowerko}% Prüfer der Arbeit
\newcommand{\dcadvisor}{M. Sc. Arunodhayan Sampath-Kumar}% Betreuer der Arbeit

\newcommand{\dckeywords}{System Design, User-centered Application, Indoor Sound Classification, Web-based Demonstration Platform, Asynchronous Messaging, Acoustic Event Classification, Decoupled Architecture, MQTT, D3.js, Real-time Streaming}

%%======================================================================
% Settings of Hyperref-Package
\hypersetup{%    
    pdftitle	= {\dctitle}, %
    pdfsubject	= {\dcsubject}, %
    pdfauthor	= {\dcauthorfirstname~\dcauthorlastname, \dcauthoremail}, %
    %	pdfkeywords	= {\dckeywords}, %
    pdfcreator	= {pdfTeX with hyperref and Thumbpdf}, %
}
%%======================================================================
