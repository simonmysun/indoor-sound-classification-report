\chapter{Summary and Outlook}
\section{Summary of Findings}
% - Summarize the key findings from the evaluation and results.
\subsection{System Overview and Achievements}
% - Recapitulate the architecture and features of your backend and frontend development, emphasizing the integration of various technologies like Docker, Prometheus, Grafana, etc.
% - Highlight the innovative aspects of your acoustic event classification web application.

This thesis introduced a novel system combining advanced acoustic event classification with a robust and scalable web-based application. The backend, built with a suite of modern technologies including \textsc{Docker}, \textsc{Prometheus}, and \textsc{Grafana}, supports comprehensive device and alert management. The frontend, a single-page application, offers real-time data streaming and intuitive data visualizations. Key achievements include the successful integration of these diverse technologies into a cohesive and efficient system.

\subsection{Performance and Reliability Insights}
% - Summarize the results of your stress tests on the MQTT server and frontend visualization, emphasizing the system's capability to handle massive streaming data.
% - Discuss the uninterrupted operational performance over several months, underlining the system's reliability and robustness.

Extensive stress testing demonstrated the system's ability to manage significant data streams without compromising performance. Notably, the uninterrupted operational performance over several months attests to the system's reliability and robustness, crucial in real-world applications of acoustic event classification.

\section{Contributions}
% - Discuss the implications of your research in the context of acoustic event classification models.
\subsection{Technical Contributions}
% - Detail how your system advances the field of acoustic event classification and web-based applications, particularly in terms of scalability, robustness, and usability.
% - Discuss the novel aspects of your backend server, MQTT-exporter, and the unique approach to data visualization in the web application.

This work significantly advances the field of acoustic event classification, especially in handling real-time data and user-friendly web interfaces. The backend’s architecture and the MQTT-exporter’s novel approach represent a substantial contribution to the field. The heatmap visualization in the frontend, depicting classification confidence, is a unique feature that enhances user experience and data interpretation.

\subsection{Future Research Implications}
% - Reflect on how your work can influence future developments in this area, potentially paving the way for more advanced implementations and research.

The system sets a precedent for future research, particularly in integrating complex backend operations with user-centric frontends. It opens avenues for more sophisticated event classification algorithms and further exploration into efficient data handling and visualization techniques.

\section{Future Work}
% - Suggest possible future research or improvements in the field.

\subsection{Proposed Enhancements}
% - Propose potential enhancements for the system, like incorporating advanced machine learning algorithms for more accurate classifications, improving user interface design, or integrating additional features based on hypothetical user feedback.

Future work could focus on integrating more advanced machine learning models for enhanced classification accuracy. Improvements in UI/UX design, based on anticipated user feedback, would make the application more intuitive and accessible. Additionally, expanding the system’s capabilities to handle a broader range of acoustic events could widen its applicability.

\subsection{Broader Applications}
% - Suggest possible extensions of your system to other domains or applications where acoustic event classification could be beneficial, such as environmental monitoring or smart city infrastructures.

Exploring the application of this system in areas like environmental monitoring or urban planning could demonstrate its versatility and potential for societal impact.

\subsection{Addressing the Lack of User Feedback}
% - Discuss strategies for obtaining user feedback in the future, such as user testing sessions or deploying the system in a real-world environment to gather practical insights.

Future deployments should include strategies to collect user feedback, such as through beta testing or pilot studies in real-world environments. This feedback is crucial for iterative improvements and ensuring the system meets actual user needs.

\section{Conclusion}
% - Provide a concluding statement that reflects on the accomplishments of your thesis.

\subsection{Reflective Overview}
% - Reflect on your experience throughout the project, acknowledging challenges faced and how they were overcome.

This journey, from conceptualization to implementation, presented numerous challenges, notably in integrating various technologies into a seamless system. Overcoming these hurdles has been a rewarding experience, contributing valuable knowledge to the field.

\subsection{Final Assessment}
% - Provide a final assessment of the project, considering its successes and areas for improvement.

The system stands as a testament to the potential of modern web technologies in enhancing the field of acoustic event classification. While there are areas for improvement, its successes in robustness, scalability, and user-friendly design are significant.

\subsection{Inspiring Future Research}
% - Offer advice or inspiration for future researchers who may want to build upon your work, encouraging them to explore uncharted areas or address unresolved challenges.

We encourage future researchers to build upon this foundation, exploring uncharted territories and addressing the challenges highlighted. There lies great potential in this field, awaiting exploration and innovation.