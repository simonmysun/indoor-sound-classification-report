\chapter{Fundamentals}
\section{Introduction to Acoustic Event Classification}
% - Describe the fundamentals of acoustic event classification models, including key concepts and technologies.

Acoustic event classification represents a significant facet of modern auditory processing, harnessing the power of deep neural networks (DNNs) for the identification and categorization of diverse sound events. This field has gained significant momentum due to its wide array of applications, ranging from environmental monitoring to urban sound analysis. A seminal work in this domain, \cite{sampath2020low}, demonstrates the efficacy of DNNs in achieving high accuracy in acoustic signal classification, underscoring the potential of these models in complex auditory environments. Complementing this, \cite{sampath2019realtime} showcases a web-based implementation of acoustic event classification, offering customizable alert functions and illustrating the practical applicability of these models. Together, these studies provide a foundational understanding of how advanced machine learning techniques, particularly deep learning, are revolutionizing the way we analyze and interpret sound data, forming a critical part of our research framework.

\section{Web-Based Solutions}
% - Explain the role of web-based solutions in enhancing accessibility and usability.
% - Discuss their importance in your research.

In the landscape of web-based solutions for deep learning applications, there exists a diverse array of tools and platforms that cater to live data streaming and alert management, which are crucial for acoustic event classification systems. Prominent among these are cloud-based solutions like AWS Timestream\cite{AWSTimestream} and Azure Cosmos DB\cite{AzureCosmos}, which have established themselves as leaders for general-purpose live data streaming. These platforms offer scalable and efficient handling of streaming data, an essential feature for real-time acoustic event analysis. On the other hand, tools like \textsc{Prometheus} \textsc{Alertmanager}\cite{AlertManager} and Kapacitor with InfluxDB\cite{kapacitor} provide advanced rule-based notifications, although they require complex configurations that can pose challenges in deployment. Notably, there is a gap in the market for on-premise, specialized design and implementation of acoustic event visualization and alarm management. This absence underscores the uniqueness and potential impact of our research, which seeks to bridge this gap by developing a tailored, web-based solution for acoustic event classification that is both accessible and efficient."

\section{User-Friendly Interfaces}
% - Explore the importance of user-friendly interfaces in machine learning applications.

The design of user-friendly interfaces plays a pivotal role in determining their accessibility and overall effectiveness in the domain of deep learning applications. This concept is particularly relevant in the context of systems like acoustic event classifiers, where the complexity of underlying technologies necessitates interfaces that are intuitive and approachable for users of all levels of technical proficiency. The principles of accessibility, usability, and inclusion, as outlined by the World Wide Web Consortium (W3C), provide a comprehensive framework for designing interfaces that accommodate a diverse range of users (\cite{w3AccessibilityUsability}). Moreover, the insights provided by \cite{SeniorfriendlyTechnologies} on designing senior-friendly technologies underscore the importance of creating interfaces that are adaptable to the needs of senior users, a demographic often overlooked in technology design. These guidelines and insights are integral to our approach, ensuring that our acoustic event classification system is not only powerful in its technical capabilities but also accessible and usable across a broad spectrum of users.