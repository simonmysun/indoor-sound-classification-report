Structured Abstract:

* should contain:

- 10 - 20 pages with real content

- 20 - 30 literature references, Note: during writing full Thesis there should be more than 30 references

- subsections/headings: not too general but precise and focused to your topic

- 1-3 bullet points per chapter and per subsections of what you will write about here

- some key Tables and Figures (can also be incomplete at this point and be completed later) including a detailed description

- 3-fold section structure, meaning chapters --> sections --> subsections (and occasionally subsubsections)




* should NOT contain

- full sentences or paragraphs

- full written chapters

- complete references of Thesis

- all figures but some

- more than 20 pages




Structure of Thesis:

Abstract / Aufgabenstellung:

contains a coarse and a detailed description of your Thesis Task




1. Introduction

- Guides into the topic taking a more general perspective of comparable themes to

      a) existing in scientific literature, communities

      b) industrial or company-related topics

- Explains the relevance of the topics, what is the technological state if the art and where are problems and challenges in this field

- may include also historical evolution of specific requirements, topics and fields

NOTE: if thesis is done in a company: briefly include the company here with not more than 3-5 main facts with focus on the part where you are working

- focuses on the own Aufgabenstellung in the end

--> all issues are already supplemented by references

- at the end of 1. an overview of the structure and the coarse content of the thesis will be given (1-3 sentences for each chapter)




2. Fundamentals and Basics

- discusses the special know-how required for this Thesis/topic

- it should not be too general and focus on state of the art hardware, software/frameworks, technology, methods/algorithms and implementations

NOTE: If thesis is done in a company: often software or environment/database is already existing and should be described here (name, version, structure, programming language, interfaces, libriaries, functionality)

- historical developments should be mentioned, e.g. when did this field emerge first, how was it solved then, later and today

--> Here should be most of the literature references relevant for the topic and reflecting the state of the art

3. Implementation
- Create a repository to manage, comment and save your code: e.g. gitlab.hrz.tu-chemnitz.de/

- analysis of requirements:

   - what functionality is already there?

   - what functionality is needed and fixed in this work?

   - hardware requirements (CPU, RAM, storage, network, offline/online, real time?, parallelisation ..., camera, microphone,...)

   - how many data are there and how many need to be processed in which period of time?

- overview of software tools and frameworks to fullfull requirements and focus THE ONES YOU USED and WHY they were used (includes you chosen programming language)

- description of your implementation

   - software patterns (model-view-controller)

   - data structure, database

   - algorithms

    - schematics/workflow diagrams

    - source code description

        - number of code lines

        - number of documentation lines

        - number of functions and/or classes, libriaries

        - Graphical User Interfaces

        NOTE: It must NOT be MANY, it should reflect to be structured and efficient

4. Results and Evaluation (can also be seperated in two chapters 4. Results of ..., 5. Evaluation of ...)


Introduction to your dataset/database
for each dataset a suitable short name

Representative Image Collage e.g. thumnails 7 lines by 7 columns representative images (for videos, use images from beginning to end every (Ntotal/50)th frames

goal of dataset/subsets: why did you chose them, what makes it special

Contents: what is in the data, describe with words or tables what is contained in the data, also histograms of occurring objects, persons, actions, defects, diseases help understanding the dataset

Number of files, file formats (raw and processed/converted/resized), (compression), resolution, frame rate (for videos), color/greyscale

description of filename, foldernames for effective Organisation/data management
Annotations: manual or automatic, annotation workflow, Annotation tool/algorithm (model driven annotation), annotation file format/convention, number of annotations (files, objects), format/convention (e.g. table/header, hierarchical (like xml, json) field description),
date (when or in which context were the data created), external --> cite ressource, e.g. publication of dataset otherwise file link
owner/rights/rights/permission for dataset 


    -


- analysis of data or workflows
- evaluation of your method:

   - may require definition of evaluation measures (recall, precision, accuracy, computation time, ...)

    - functionality comparison of the state before and after implementation

     Table with

      category, yes/no

      #1            yes

---> Tables and graphs should visually reflect the results

---> Results should be interpreted (reason for good/bad/average results)

---> Evaluations should be used to conclude on best strategies or algorithms to solve the task




5. Conclusion and Outlook

conclusion:

- summarize basically what is new for a scientific community, namely your implementation, its results and the evaluation

- discuss and interpret your results in the context of your Aufgabenstellung and the general research or developer community

outlook:

- what can be improved in your system

- what other approaches are there