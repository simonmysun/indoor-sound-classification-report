%======================================================================
%	Title Page
%======================================================================
%	$Date:$
%	$Revision:$
%	Matthias Kupfer
%======================================================================

%%======================================================================
%% Extra Title
%%======================================================================
%\extratitle{
%	\usekomafont{disposition}\mdseries 
%	\begin{center}
%		\Huge \dcsubject\\[1.5ex]
%		\hrule
%		\vspace*{\fill}
%		\includegraphics{TUC_deutsch_einzeile_CMYK}
%	\end{center}
%}

%%======================================================================
%% Title Head
%%======================================================================
\titlehead{
  \vspace*{-1.5cm}
  % font families like all headings, but not bold
  \usekomafont{disposition}\mdseries
  \begin{center}
    \raisebox{-1ex}{\includegraphics[scale=1.4]{TUC_deutsch_einzeile_CMYK}}\\
    \hrulefill \\[1em]
    {\Large\dcdepart}\\[0.5em]
    \dcprof
  \end{center}
  \vspace*{1.5cm}
}

%%======================================================================
%% Subject
%%======================================================================


\subject{\normalfont\bfseries\Huge\dcsubject}

%%======================================================================
%% Title
%%======================================================================
\title{\normalfont\sffamily\Large
  \dctitle
  \\
  \dcsubtitle
}

%%======================================================================
%% Author of Document
%%======================================================================
\author{\dcauthorfirstname~\dcauthorlastname}

%%======================================================================
%% Place, Date
%%======================================================================
\date{\dcplace, \dcdate}

%%======================================================================
%% Publishers
%%======================================================================
\publishers{
  {\parbox{\textwidth-8em}{
        \begin{tabbing}
          {\normalfont\bfseries Supervisor:}\quad\=\kill
          {\normalfont\bfseries Examiner:}	\>\dcpruefer\\
          {\normalfont\bfseries Supervisor:}	\>\dcadvisor
        \end{tabbing}
      }}
}

%%======================================================================
%% Bibliographic Statement
%%======================================================================
\lowertitleback{
  \textbf{\dcauthorlastname, \dcauthorfirstname}\\
  \dctitle\\
  \dcsubject,~\dcdepart\\
  \dcuni,~\ifcase\month\or
    January\or February\or March\or April\or May\or June\or
    July\or August\or September\or October\or November\or December\fi
  ~\number\year
}

%%======================================================================
%% maketitle
%%======================================================================

\maketitle



%%======================================================================
%% Acknowledgement
%%======================================================================
\thispagestyle{empty}
%\null\vfil
\begin{center}
  \usekomafont{disposition}
  \textbf{{Acknowledgement}}
  %\vspace{-.5em}\vspace{\parsep}
\end{center}

I would like to express my sincere gratitude to a number of people whose support and guidance have been invaluable throughout the journey of this thesis.

First and foremost, my deepest appreciation goes to my thesis advisors, Prof. Danny Kowerko and Mr. Arunodhayan Sampath-Kumar, whose expertise, understanding, and patience, added considerably to my graduate experience. Your guidance was invaluable at every stage of this research, and I am immensely grateful for your input and encouragement.

I also wish to express my appreciation to the faculty and staff in the Junior Professorship of Media Computing at TU Chemnitz. Your dedication to fostering a rigorous academic environment has profoundly enriched my educational experience.

I would also like to acknowledge the technical support and resources provided by our university, which were crucial to the successful completion of this project.

To my family and my girlfriend, thank you for your unwavering support and encouragement throughout my years of study and through the process of researching and writing this thesis. This accomplishment would not have been possible without you.

Special thanks are due to Ms. Liang for her assistance in printing this thesis and submitting it to the examination office, without whom, I wouldn't be able to submit it on time.

Lastly, I cannot forget to express my gratitude to all those who were involved in the practical aspects of this project. Your contributions have been immensely valuable.

Thank you all for your support and encouragement.

\let\cleardoublepage\cleardoubleemptypage
%\cleardoublepage

%%======================================================================
%%       Abstract
%%======================================================================
\begin{abstract}
  The increasing demand for robust and efficient acoustic event classification models necessitates innovative approaches to their demonstration and deployment. This thesis presents a novel web-based demonstration platform designed to showcase the capabilities of an acoustic event classification model proposed by \cite{sampath2019cnn}. Emphasizing user-friendliness, configurability, scalability, and robustness, the implementation leverages asynchronous messaging and a decoupled architecture, utilizing MQTT as the messaging protocol for the backend and \textsc{d3.js} for visualization in the frontend.

  The user-friendly interface allows users to interact intuitively with the acoustic event classification model, facilitating a seamless experience. The system's configurability empowers users to adapt the demonstration to various scenarios and applications. Scalability is achieved through a decoupled architecture that ensures efficient resource utilization and accommodates varying workloads. The thesis explores the integration of asynchronous messaging to enhance the responsiveness and real-time capabilities of the demonstration. By decoupling components, the architecture not only improves scalability but also contributes to the system's robustness, ensuring reliable performance under diverse conditions. Throughout the thesis, we detail the implementation and deployment of this demonstration platform, emphasizing the technical considerations of the asynchronous messaging system and the advantages of a decoupled architecture. Evaluation results demonstrate the platform's efficiency, highlighting its potential for broader applications in acoustic event classification.

  This research contributes to the intersection of acoustic event classification, web-based demonstrations, and system architecture, offering a valuable resource for researchers, practitioners, and developers in the field. The demonstrated platform serves as a benchmark for user-friendly and scalable acoustic event classification models, paving the way for advancements in real-world applications.

\end{abstract}

%%======================================================================
%%      Table of Contents
%%======================================================================
%\cleardoubleemptypage
\pdfbookmark{Contents}{Contents}
\pagenumbering{roman}
\tableofcontents

%%======================================================================
%%      List of Figures
%%======================================================================
%\cleardoublepage
%\markboth{List of Figures}{List of Figures}
\listoffigures

%%======================================================================
%%      List of Tables
%%======================================================================
%\cleardoublepage
%\markboth{List of Tables}{List of Tables}
\listoftables

%%======================================================================
%%      List of Algorithms
%%======================================================================
%\renewcommand{\listalgorithmname}{List of Algorithms}
%\cleardoublepage
%\addcontentsline{toc}{chapter}{List of Algorithms}
%\listofalgorithms

%%======================================================================
%%      Glossary
%%======================================================================
%\cleardoublepage               

\newacronym{acronym1}{ACR1}{Acronym 1}
\newacronym{acronym2}{ACR2}{Acronym 2}
\newacronym{acronym3}{ACR3}{Acronym 3}

\printglossary
\printglossary[type=\acronymtype,title=List of Acronyms,toctitle=List of Acronyms]

%%======================================================================
%%      End
%%======================================================================
%\cleardoublepage
%\pagenumbering{arabic}

