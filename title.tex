%======================================================================
%	Title Page
%======================================================================
%	$Date:$
%	$Revision:$
%	Matthias Kupfer
%======================================================================

%%======================================================================
%% Extra Title
%%======================================================================
%\extratitle{
%	\usekomafont{disposition}\mdseries 
%	\begin{center}
%		\Huge \dcsubject\\[1.5ex]
%		\hrule
%		\vspace*{\fill}
%		\includegraphics{TUC_deutsch_einzeile_CMYK}
%	\end{center}
%}

%%======================================================================
%% Title Head
%%======================================================================
\titlehead{
  \vspace*{-1.5cm}
  % font families like all headings, but not bold
  \usekomafont{disposition}\mdseries
  \begin{center}
    \raisebox{-1ex}{\includegraphics[scale=1.4]{TUC_deutsch_einzeile_CMYK}}\\
    \hrulefill \\[1em]
    {\Large\dcdepart}\\[0.5em]
    \dcprof
  \end{center}
  \vspace*{1.5cm}
}

%%======================================================================
%% Subject
%%======================================================================


\subject{\normalfont\bfseries\Huge\dcsubject}

%%======================================================================
%% Title
%%======================================================================
\title{\normalfont\sffamily\Large
  \dctitle
  \\
  \dcsubtitle
}

%%======================================================================
%% Author of Document
%%======================================================================
\author{\dcauthorfirstname~\dcauthorlastname}

%%======================================================================
%% Place, Date
%%======================================================================
\date{\dcplace, \dcdate}

%%======================================================================
%% Publishers
%%======================================================================
\publishers{
  {\parbox{\textwidth-8em}{
        \begin{tabbing}
          {\normalfont\bfseries Supervisor:}\quad\=\kill
          {\normalfont\bfseries Examiner:}	\>\dcpruefer\\
          {\normalfont\bfseries Supervisor:}	\>\dcadvisor
        \end{tabbing}
      }}
}

%%======================================================================
%% Bibliographic Statement
%%======================================================================
\lowertitleback{
  \textbf{\dcauthorlastname, \dcauthorfirstname}\\
  \dctitle\\
  \dcsubject,~\dcdepart\\
  \dcuni,~\ifcase\month\or
    January\or February\or March\or April\or May\or June\or
    July\or August\or September\or October\or November\or December\fi
  ~\number\year
}

%%======================================================================
%% maketitle
%%======================================================================

\maketitle



%%======================================================================
%% Acknowledgement
%%======================================================================
\thispagestyle{empty}
%\null\vfil
\begin{center}
  \usekomafont{disposition}
  \textbf{{Acknowledgement}}
  %\vspace{-.5em}\vspace{\parsep}
\end{center}

I would like to express my sincere gratitude to a number of people whose support and guidance have been invaluable throughout the journey of this thesis.

First and foremost, my deepest appreciation goes to my thesis advisors, Prof. Danny Kowerko and Mr. Arunodhayan Sampath-Kumar, whose expertise, understanding, and patience, added considerably to my graduate experience. Your guidance was invaluable at every stage of this research, and I am immensely grateful for your input and encouragement.

I also wish to express my appreciation to the faculty and staff in the Junior Professorship of Media Computing at TU Chemnitz. Your dedication to fostering a rigorous academic environment has profoundly enriched my educational experience.

I would also like to acknowledge the technical support and resources provided by our university, which were crucial to the successful completion of this project.

To my family, thank you for your unwavering support and encouragement throughout my years of study and through the process of researching and writing this thesis. This accomplishment would not have been possible without you.

Lastly, I cannot forget to express my gratitude to all those who were involved in the practical aspects of this project. Your contributions have been immensely valuable.

Thank you all for your support and encouragement.

\let\cleardoublepage\cleardoubleemptypage
%\cleardoublepage

%%======================================================================
%%       Abstract
%%======================================================================
\begin{abstract}
  In this thesis, we introduce a comprehensive web-based solution designed to harness the capabilities of the acoustic event classification model proposed by \cite{sampath2019cnn}. Our system integrates a robust and scalable backend with a user-friendly frontend, offering a seamless experience in visualizing and managing acoustic data. The backend architecture encompasses a server handling device and alert management, an MQTT-exporter to record classifier results in \textsc{Prometheus} metric format, and a suite of supporting technologies including \textsc{Docker Compose}, \textsc{Eclipse Mosquitto}, \textsc{Prometheus}, \textsc{Grafana}, \textsc{Alertmanager}, \textsc{Nginx}, and \textsc{Traefik}. The frontend is a single-page application providing a live view of data streamed from the acoustic event classifier, rendered in a heatmap form with classification tags and prediction confidence levels. Additional features include interfaces for device and alert management, as well as historical data visualizations through \textsc{Prometheus} time-series database queries via \textsc{Grafana}, and a settings page for server credential management. Our evaluation, comprising stress tests on the MQTT server and frontend visualization rendering with massive streaming data, demonstrates the system's robustness and efficiency, evidenced by several months of uninterrupted operation. This work not only leverages the acoustic event classification model but also advances the field by providing a holistic, scalable, and user-friendly platform for acoustic data analysis and alert management, offering significant contributions to real-time monitoring and decision-making processes in various application domains.
\end{abstract}

%%======================================================================
%%      Table of Contents
%%======================================================================
%\cleardoubleemptypage
\pdfbookmark{Contents}{Contents}
\pagenumbering{roman}
\tableofcontents

%%======================================================================
%%      List of Figures
%%======================================================================
%\cleardoublepage
%\markboth{List of Figures}{List of Figures}
\listoffigures

%%======================================================================
%%      List of Tables
%%======================================================================
%\cleardoublepage
%\markboth{List of Tables}{List of Tables}
\listoftables

%%======================================================================
%%      List of Algorithms
%%======================================================================
%\renewcommand{\listalgorithmname}{List of Algorithms}
%\cleardoublepage
%\addcontentsline{toc}{chapter}{List of Algorithms}
%\listofalgorithms

%%======================================================================
%%      Glossary
%%======================================================================
%\cleardoublepage               

\newacronym{acronym1}{ACR1}{Acronym 1}
\newacronym{acronym2}{ACR2}{Acronym 2}
\newacronym{acronym3}{ACR3}{Acronym 3}

\printglossary
\printglossary[type=\acronymtype,title=List of Acronyms,toctitle=List of Acronyms]

%%======================================================================
%%      End
%%======================================================================
%\cleardoublepage
%\pagenumbering{arabic}

